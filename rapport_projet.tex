\documentclass[12pt,a4paper]{article}
\usepackage[utf8]{inputenc}
\usepackage[french]{babel}
\usepackage[T1]{fontenc}
\usepackage{float}
\usepackage{geometry}
\usepackage{graphicx}
\usepackage{hyperref}
\usepackage{listings}
\usepackage{xcolor}
\usepackage{fancyhdr}
\usepackage{titlesec}
\usepackage{enumitem}
\usepackage{booktabs}
\usepackage{longtable}
\usepackage{amsmath}
\usepackage{amsfonts}
\usepackage{amssymb}

% Configuration de la page
\geometry{left=2.5cm,right=2.5cm,top=2.5cm,bottom=2.5cm}

% Configuration des en-têtes et pieds de page
\pagestyle{fancy}
\fancyhf{}
\fancyhead[L]{Système de Réservation de Rendez-vous}
\fancyhead[R]{Experlik}
\fancyfoot[C]{\thepage}

% Configuration des couleurs
\definecolor{codegreen}{rgb}{0,0.6,0}
\definecolor{codegray}{rgb}{0.5,0.5,0.5}
\definecolor{codepurple}{rgb}{0.58,0,0.82}
\definecolor{backcolour}{rgb}{0.95,0.95,0.92}

% Configuration du style de code
\lstdefinestyle{mystyle}{
    backgroundcolor=\color{backcolour},   
    commentstyle=\color{codegreen},
    keywordstyle=\color{magenta},
    numberstyle=\tiny\color{codegray},
    stringstyle=\color{codepurple},
    basicstyle=\ttfamily\footnotesize,
    breakatwhitespace=false,         
    breaklines=true,                 
    captionpos=b,                    
    keepspaces=true,                 
    numbers=left,                    
    numbersep=5pt,                  
    showspaces=false,                
    showstringspaces=false,
    showtabs=false,                  
    tabsize=2
}

\lstset{style=mystyle}

% Configuration des titres
\titleformat{\section}{\Large\bfseries\color{blue!70!black}}{\thesection}{1em}{}
\titleformat{\subsection}{\large\bfseries\color{blue!50!black}}{\thesubsection}{1em}{}

\begin{document}

% Page de titre
\begin{titlepage}
    \centering
    \vspace*{2cm}
    
    {\Huge\bfseries Système de Réservation de Rendez-vous}\\[0.5cm]
    {\Large Plateforme Web Complète}\\[1.5cm]
    
    {\large Projet développé par Experlik}\\[2cm]
    
    \begin{tabular}{ll}
        \textbf{Technologies utilisées:} & React.js, Node.js, MongoDB \\
        \textbf{Déploiement:} & Vercel \\
        \textbf{Date:} & \today \\
    \end{tabular}
    
    \vfill
    
    {\large Rapport technique complet}
    
\end{titlepage}

% Table des matières
\tableofcontents
\newpage

\section{Introduction}

Ce rapport présente un système complet de réservation de rendez-vous développé par Experlik. La plateforme permet aux utilisateurs de réserver des rendez-vous avec différents prestataires de services, de gérer des événements, et d'utiliser des codes promotionnels.

\subsection{Objectifs du projet}

\begin{itemize}
    \item Créer une plateforme intuitive pour la réservation de rendez-vous
    \item Permettre aux prestataires de services de gérer leur planning
    \item Intégrer des systèmes de paiement multiples
    \item Offrir un système de gestion d'événements
    \item Fournir des outils d'administration complets
\end{itemize}

\subsection{Portée du projet}

\begin{figure}[H]
    \centering
    \includegraphics[width=0.9\textwidth]{diagrammes/use_case_diagram.png}
    \caption{Diagramme de cas d'utilisation du système}
\end{figure}

Le système comprend trois interfaces principales :
\begin{enumerate}
    \item Interface utilisateur (frontend)
    \item Interface d'administration (admin)
    \item API backend avec base de données
\end{enumerate}

\section{Architecture du Système}

\subsection{Vue d'ensemble}

\begin{figure}[H]
    \centering
    \includegraphics[width=0.9\textwidth]{diagrammes/architecture_diagram.png}
    \caption{Architecture générale du système}
\end{figure}

Le projet suit une architecture moderne en trois couches :

\begin{itemize}
    \item \textbf{Frontend} : Application React.js avec Vite
    \item \textbf{Backend} : API REST avec Node.js et Express
    \item \textbf{Base de données} : MongoDB avec Mongoose ODM
\end{itemize}

\subsection{Structure des dossiers}

\begin{lstlisting}[language=bash, caption=Structure du projet]
CalendlyClone/
├── frontend/          # Interface utilisateur
├── admin/            # Interface d'administration
├── backend/          # API et logique métier
└── README.md         # Documentation
\end{lstlisting}

\section{Technologies Utilisées}

\subsection{Frontend}

\begin{longtable}{|l|l|p{8cm}|}
\hline
\textbf{Technologie} & \textbf{Version} & \textbf{Utilisation} \\
\hline
React & 19.1.0 & Framework principal pour l'interface utilisateur \\
\hline
Vite & 7.0.0 & Outil de build et serveur de développement \\
\hline
React Router & 7.6.3 & Gestion de la navigation \\
\hline
Axios & 1.10.0 & Client HTTP pour les appels API \\
\hline
Tailwind CSS & 3.4.17 & Framework CSS pour le styling \\
\hline
React Toastify & 11.0.5 & Notifications utilisateur \\
\hline
React Icons & 5.5.0 & Bibliothèque d'icônes \\
\hline
\end{longtable}

\subsection{Backend}

\begin{longtable}{|l|l|p{8cm}|}
\hline
\textbf{Technologie} & \textbf{Version} & \textbf{Utilisation} \\
\hline
Node.js & - & Runtime JavaScript côté serveur \\
\hline
Express & 5.1.0 & Framework web pour Node.js \\
\hline
MongoDB & - & Base de données NoSQL \\
\hline
Mongoose & 8.16.1 & ODM pour MongoDB \\
\hline
JWT & 9.0.2 & Authentification par tokens \\
\hline
Bcrypt & 6.0.0 & Hachage des mots de passe \\
\hline
Cloudinary & 2.7.0 & Stockage et gestion d'images \\
\hline
Stripe & 18.3.0 & Système de paiement \\
\hline
Razorpay & 2.9.6 & Système de paiement alternatif \\
\hline
Nodemailer & 7.0.5 & Envoi d'emails \\
\hline
\end{longtable}

\section{Fonctionnalités Principales}

\subsection{Gestion des Utilisateurs}

\subsubsection{Inscription et Connexion}
\begin{itemize}
    \item Inscription avec email et mot de passe
    \item Connexion sécurisée avec JWT
    \item Gestion des profils utilisateur
    \item Upload d'images de profil via Cloudinary
\end{itemize}

\subsubsection{Profils Utilisateur}
\begin{itemize}
    \item Informations personnelles complètes
    \item Gestion des adresses
    \item Numéro de téléphone avec validation internationale
    \item Historique des rendez-vous
\end{itemize}

\subsection{Système de Réservation}

\subsubsection{Réservation Classique}
\begin{itemize}
    \item Sélection de prestataires par spécialité
    \item Choix de créneaux horaires disponibles
    \item Confirmation de rendez-vous
    \item Notifications par email
\end{itemize}

\subsubsection{Système de Créneaux Calendrier}
\begin{itemize}
    \item Configuration flexible des horaires de travail
    \item Génération automatique de créneaux
    \item Gestion des pauses et jours non travaillés
    \item Réservation en temps réel
\end{itemize}

\subsection{Gestion des Prestataires}

\subsubsection{Inscription des Prestataires}
\begin{itemize}
    \item Processus d'inscription en deux étapes
    \item Vérification par OTP email
    \item Validation des informations professionnelles
    \item Approbation par l'administrateur
\end{itemize}

\subsubsection{Interface Prestataire}
\begin{itemize}
    \item Dashboard avec statistiques
    \item Gestion des rendez-vous
    \item Configuration du calendrier
    \item Gestion des événements
    \item Système de codes promotionnels
\end{itemize}

\subsection{Système d'Événements}

\subsubsection{Création d'Événements}
\begin{itemize}
    \item Événements gratuits ou payants
    \item Gestion des participants
    \item Dates limites d'inscription
    \item Localisation physique ou en ligne
\end{itemize}

\subsubsection{Inscription aux Événements}
\begin{itemize}
    \item Inscription simple pour événements gratuits
    \item Paiement intégré pour événements payants
    \item Application de codes promotionnels
    \item Confirmation par email
\end{itemize}

\subsection{Système de Paiement}

Le système intègre trois solutions de paiement :

\subsubsection{Stripe}
\begin{itemize}
    \item Paiements par carte bancaire
    \item Interface sécurisée
    \item Gestion des remboursements
\end{itemize}

\subsubsection{Razorpay}
\begin{itemize}
    \item Solution de paiement alternative
    \item Support de multiples devises
    \item Interface utilisateur optimisée
\end{itemize}

\subsubsection{Payzone}
\begin{itemize}
    \item Solution de paiement locale
    \item Intégration personnalisée
    \item Redirection sécurisée
\end{itemize}

\section{Base de Données}

\begin{figure}[H]
    \centering
    \includegraphics[width=0.9\textwidth]{diagrammes/class_diagram.png}
    \caption{Diagramme de classes - Modèles de données}
\end{figure}

\subsection{Relations entre les entités}

Le diagramme de classes ci-dessus illustre les principales entités du système et leurs relations. Les modèles principaux incluent les utilisateurs, les prestataires de services, les rendez-vous, les événements, et les codes promotionnels.

\subsection{Modèles de Données}

\subsubsection{Modèle Utilisateur}
\begin{lstlisting}[language=JavaScript, caption=Schéma utilisateur]
const userSchema = new mongoose.Schema({
    name: {type: String, required: true},
    email: {type: String, required: true, unique: true},
    password: {type: String, required: true},
    image: {type: String, default: "..."},
    address: {type: Object, default: {line1:'', line2:''}},
    gender: {type: String, default: "Not Selected"},
    dob: {type: String, default: "Not Selected"},
    phone: {type: String, default: "000000000"}
})
\end{lstlisting}

\subsubsection{Modèle Prestataire}
\begin{lstlisting}[language=JavaScript, caption=Schéma prestataire]
const serviceSchema = new mongoose.Schema({
    name: {type: String, required: true},
    email: {type: String, required: true, unique: true},
    password: {type: String, required: true},
    image: {type: String, required: true},
    speciality: {type: String, required: true},
    degree: {type: String, required: true},
    experience: {type: String, required: true},
    about: {type: String, required: true},
    available: {type: Boolean, default:true},
    fees: {type: Number, required: true},
    address: {type: Object, required: true},
    slots_booked: {type: Object, default: {}}
})
\end{lstlisting}

\subsubsection{Modèle Rendez-vous}
\begin{lstlisting}[language=JavaScript, caption=Schéma rendez-vous]
const appointmentSchema = new mongoose.Schema({
    userId: { type: String, required: true },
    docId: { type: String, required: true },
    slotDate: { type: String, required: true },
    slotTime: { type: String, required: true },
    userData: { type: Object, required: true },
    docData: { type: Object, required: true },
    amount: { type: Number, required: true },
    date: { type: Number, required: true },
    cancelled: { type: Boolean, default: false },
    payment: { type: Boolean, default: false },
    isCompleted: { type: Boolean, default: false }
});
\end{lstlisting}

\subsection{Modèles Avancés}

\subsubsection{Événements}
\begin{lstlisting}[language=JavaScript, caption=Schéma événement]
const eventSchema = new mongoose.Schema({
    title: { type: String, required: true },
    description: { type: String, required: true },
    location: { type: String, required: true },
    startDate: { type: Date, required: true },
    endDate: { type: Date, required: true },
    maxParticipants: { type: Number, required: true },
    registrationDeadline: { type: Date, required: true },
    isFree: { type: Boolean, default: true },
    price: { type: Number, default: 0 },
    serviceId: { type: String, required: true },
    participants: [{ 
        userId: String,
        userData: Object,
        registrationDate: { type: Date, default: Date.now },
        promoCode: String,
        finalPrice: Number
    }]
});
\end{lstlisting}

\subsubsection{Codes Promotionnels}
\begin{lstlisting}[language=JavaScript, caption=Schéma code promo]
const promoCodeSchema = new mongoose.Schema({
    code: { type: String, required: true, unique: true },
    serviceId: { type: String, required: true },
    discountType: { type: String, enum: ['fixed', 'percentage'] },
    discountValue: { type: Number, required: true },
    usageLimit: { type: Number, default: null },
    usagePerUser: { type: Number, default: 1 },
    validFrom: { type: Date, required: true },
    validUntil: { type: Date, required: true },
    usedBy: [{ 
        userId: String,
        usageCount: { type: Number, default: 0 },
        lastUsed: Date
    }],
    totalUsed: { type: Number, default: 0 },
    isActive: { type: Boolean, default: true }
});
\end{lstlisting}

\section{API et Endpoints}

\subsection{Routes Utilisateur}

\begin{longtable}{|l|l|p{6cm}|}
\hline
\textbf{Méthode} & \textbf{Endpoint} & \textbf{Description} \\
\hline
POST & /api/user/register & Inscription utilisateur \\
\hline
POST & /api/user/login & Connexion utilisateur \\
\hline
GET & /api/user/get-profile & Récupération du profil \\
\hline
POST & /api/user/update-profile & Mise à jour du profil \\
\hline
POST & /api/user/book-appointment & Réservation de rendez-vous \\
\hline
GET & /api/user/appointments & Liste des rendez-vous \\
\hline
POST & /api/user/cancel-appointment & Annulation de rendez-vous \\
\hline
POST & /api/user/payment-stripe & Paiement Stripe \\
\hline
POST & /api/user/payment-razorpay & Paiement Razorpay \\
\hline
\end{longtable}

\subsection{Routes Prestataire}

\begin{longtable}{|l|l|p{6cm}|}
\hline
\textbf{Méthode} & \textbf{Endpoint} & \textbf{Description} \\
\hline
GET & /api/service/list & Liste des prestataires \\
\hline
POST & /api/service/login & Connexion prestataire \\
\hline
POST & /api/service/register-step1 & Inscription étape 1 \\
\hline
POST & /api/service/register-step2 & Inscription étape 2 \\
\hline
GET & /api/service/appointments & Rendez-vous du prestataire \\
\hline
GET & /api/service/dashboard & Données du tableau de bord \\
\hline
GET & /api/service/profile & Profil du prestataire \\
\hline
\end{longtable}

\subsection{Routes Administration}

\begin{longtable}{|l|l|p{6cm}|}
\hline
\textbf{Méthode} & \textbf{Endpoint} & \textbf{Description} \\
\hline
POST & /api/admin/login & Connexion administrateur \\
\hline
POST & /api/admin/add-service & Ajout de prestataire \\
\hline
POST & /api/admin/all-services & Liste des prestataires \\
\hline
GET & /api/admin/appointments & Tous les rendez-vous \\
\hline
GET & /api/admin/dashboard & Tableau de bord admin \\
\hline
\end{longtable}

\section{Processus Métier}

\subsection{Processus de Réservation}

\begin{figure}[H]
    \centering
    \includegraphics[width=0.9\textwidth]{diagrammes/activity_diagram.png}
    \caption{Diagramme d'activité - Processus de réservation}
\end{figure}

\subsection{Cycle de Vie des Rendez-vous}

\begin{figure}[H]
    \centering
    \includegraphics[width=0.9\textwidth]{diagrammes/state_diagram.png}
    \caption{Diagramme d'états - Cycle de vie d'un rendez-vous}
\end{figure}

\section{Sécurité}

\subsection{Authentification}

\subsubsection{JWT (JSON Web Tokens)}
\begin{itemize}
    \item Tokens sécurisés pour l'authentification
    \item Expiration automatique des sessions
    \item Middleware de vérification des tokens
\end{itemize}

\subsubsection{Hachage des Mots de Passe}
\begin{lstlisting}[language=JavaScript, caption=Hachage avec bcrypt]
const salt = await bcrypt.genSalt(10);
const hashedPassword = await bcrypt.hash(password, salt);
\end{lstlisting}

\subsection{Validation des Données}

\subsubsection{Validation Email}
\begin{lstlisting}[language=JavaScript, caption=Validation avec validator]
if (!validator.isEmail(email)) {
    return res.json({
        success: false, 
        message: "Please enter a valid email"
    });
}
\end{lstlisting}

\subsubsection{Validation OTP}
\begin{itemize}
    \item Codes OTP à 6 chiffres
    \item Expiration après 10 minutes
    \item Vérification par email
\end{itemize}

\subsection{Protection des Routes}

\begin{lstlisting}[language=JavaScript, caption=Middleware d'authentification]
const authUser = async (req, res, next) => {
    try {
        const {token} = req.headers
        if(!token){
            return res.json({
                success:false, 
                message:"Not authorized, Login again"
            })
        }
        const token_decode = jwt.verify(token, process.env.JWT_SECRET)
        req.userId = token_decode.id;
        next();
    } catch (error) {
        console.log(error)
        return res.json({success:false, message:error.message})
    }
}
\end{lstlisting}

\section{Interface Utilisateur}

\subsection{Design et UX}

\subsubsection{Principes de Design}
\begin{itemize}
    \item Design responsive avec Tailwind CSS
    \item Interface intuitive et moderne
    \item Animations et transitions fluides
    \item Accessibilité optimisée
\end{itemize}

\subsubsection{Composants Principaux}
\begin{itemize}
    \item \textbf{Navbar} : Navigation principale avec menu responsive
    \item \textbf{Header} : Section d'accueil avec call-to-action
    \item \textbf{SpecialityMenu} : Sélection par spécialité avec défilement automatique
    \item \textbf{TopServices} : Affichage des meilleurs prestataires
    \item \textbf{Footer} : Informations de contact et liens sociaux
\end{itemize}

\subsection{Pages Principales}

\subsubsection{Page d'Accueil}
\begin{itemize}
    \item Présentation des services
    \item Menu des spécialités
    \item Top des prestataires
    \item Bannière d'inscription
\end{itemize}

\subsubsection{Page de Réservation}
\begin{itemize}
    \item Informations détaillées du prestataire
    \item Sélection de créneaux horaires
    \item Confirmation de rendez-vous
    \item Services similaires
\end{itemize}

\subsubsection{Profil Utilisateur}
\begin{itemize}
    \item Édition des informations personnelles
    \item Upload d'image de profil
    \item Gestion des adresses
    \item Historique des rendez-vous
\end{itemize}

\section{Déploiement}

\subsection{Environnements}

\subsubsection{URLs de Déploiement}
\begin{itemize}
    \item \textbf{Frontend} : \url{https://calendlyclone-chi.vercel.app}
    \item \textbf{Backend} : \url{https://calendlyclone-back.vercel.app}
    \item \textbf{Admin} : \url{https://calendlyclone-high.vercel.app}
\end{itemize}

\subsubsection{Configuration Vercel}
\begin{lstlisting}[language=JSON, caption=vercel.json]
{
    "rewrites": [
        {
            "source": "/(.*)",
            "destination": "/"
        }
    ]
}
\end{lstlisting}

\subsection{Variables d'Environnement}

\subsubsection{Backend}
\begin{itemize}
    \item \texttt{MONGODB\_URI} : Connexion à la base de données
    \item \texttt{JWT\_SECRET} : Clé secrète pour les tokens
    \item \texttt{CLOUDINARY\_*} : Configuration Cloudinary
    \item \texttt{STRIPE\_SECRET\_KEY} : Clé API Stripe
    \item \texttt{RAZORPAY\_*} : Configuration Razorpay
    \item \texttt{EMAIL\_*} : Configuration SMTP
\end{itemize}

\subsubsection{Frontend}
\begin{itemize}
    \item \texttt{VITE\_BACKEND\_URL} : URL de l'API backend
    \item \texttt{VITE\_RAZORPAY\_KEY\_ID} : Clé publique Razorpay
\end{itemize}

\section{Tests et Qualité}

\subsection{Tests Fonctionnels}

\subsubsection{Scénarios de Test}
\begin{enumerate}
    \item Inscription et connexion utilisateur
    \item Réservation de rendez-vous
    \item Paiement et confirmation
    \item Gestion des événements
    \item Interface d'administration
\end{enumerate}

\subsection{Optimisations}

\subsubsection{Performance}
\begin{itemize}
    \item Lazy loading des composants
    \item Optimisation des images avec Cloudinary
    \item Mise en cache des données
    \item Compression des assets
\end{itemize}

\subsubsection{SEO}
\begin{itemize}
    \item Meta tags optimisés
    \item URLs sémantiques
    \item Sitemap automatique
    \item Schema markup
\end{itemize}

\section{Maintenance et Évolutions}

\subsection{Monitoring}

\subsubsection{Logs et Erreurs}
\begin{itemize}
    \item Logging centralisé des erreurs
    \item Monitoring des performances
    \item Alertes automatiques
    \item Tableaux de bord de santé
\end{itemize}

\subsection{Évolutions Futures}

\subsubsection{Fonctionnalités Prévues}
\begin{itemize}
    \item Système de chat en temps réel
    \item Notifications push
    \item Application mobile
    \item Intégration calendrier externe
    \item Système de reviews et ratings
\end{itemize}

\subsubsection{Améliorations Techniques}
\begin{itemize}
    \item Migration vers TypeScript
    \item Tests automatisés complets
    \item CI/CD avec GitHub Actions
    \item Microservices architecture
\end{itemize}

\section{Conclusion}

\subsection{Résultats Obtenus}

Le projet a abouti à une plateforme complète et fonctionnelle offrant :

\begin{itemize}
    \item Une interface utilisateur moderne et responsive
    \item Un système de réservation flexible et intuitif
    \item Des outils d'administration complets
    \item Une intégration de paiement sécurisée
    \item Un système d'événements innovant
\end{itemize}

\subsection{Défis Relevés}

\subsubsection{Techniques}
\begin{itemize}
    \item Intégration de multiples systèmes de paiement
    \item Gestion complexe des créneaux horaires
    \item Optimisation des performances
    \item Sécurisation des données sensibles
\end{itemize}

\subsubsection{Fonctionnels}
\begin{itemize}
    \item Interface intuitive pour tous types d'utilisateurs
    \item Gestion des conflits de réservation
    \item Système de notifications efficace
    \item Évolutivité de la plateforme
\end{itemize}

\section{Interactions Système}

\subsection{Séquence de Réservation}

\begin{figure}[H]
    \centering
    \includegraphics[width=0.9\textwidth]{diagrammes/sequence_reservation.png}
    \caption{Diagramme de séquence - Réservation de rendez-vous}
\end{figure}

\subsection{Inscription des Prestataires}

\begin{figure}[H]
    \centering
    \includegraphics[width=0.9\textwidth]{diagrammes/sequence_inscription_prestataire.png}
    \caption{Diagramme de séquence - Inscription prestataire}
\end{figure}

\subsection{Impact et Valeur Ajoutée}

Cette plateforme apporte une solution complète pour :
\begin{itemize}
    \item La digitalisation des prises de rendez-vous
    \item L'optimisation de la gestion du temps
    \item La facilitation des paiements en ligne
    \item L'amélioration de l'expérience client
\end{itemize}

Le système développé constitue une base solide pour l'expansion future et l'ajout de nouvelles fonctionnalités, répondant aux besoins évolutifs du marché de la réservation en ligne.

\end{document}